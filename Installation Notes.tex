	\chapter{Installation Notes}

	\section{Virtual Environments}
From the Anaconda command prompt:
	\begin{code}[\codenumbering]{}
		\codeitemnonumber Navigate to the directory where you want the environment installed.
		\codeitemnonumber python -m venv \textit{<environmentname>}
		\codeitemnonumber cd \textit{<environmentname>}/Scripts/
		\codeitemnonumber activate.bat [from scripts directory]
		\codeitemnonumber pip install spyder-kernels==2.2.*
	\end{code}

	\section{Django}
Reference: \href{https://realpython.com/django-setup/#mysql}{Django installation}

\noindent From the Anaconda command prompt:
	\begin{code}[\codenumbering]{}
		\codeitemnonumber python -m pip install django
		\codeitemnonumber python -m pip freeze > requirements.txt
		\codeitemnonumber django-admin startproject \textit{<projectname>}
		\codeitemnonumber cd \textit{<projectname>}
		\codeitemnonumber python manage.py startapp \textit{<appname>}
		\codeitemnonumber python manage.py makemigrations \textit{<appname>} [from directory with manage.py]
		\codeitemnonumber python manage.py migrate \textit{<appname>}
	\end{code}


	\section{Miscellaneous}
\textcode{!\{sys.executable\} -m pip install pandas-profiling}

Use this to import from another directory.
	\begin{code}[\codenumbering]{}
		\codeitemnonumber import sys
		\codeitemnonumber sys.path.insert(1, "..\textbackslash{}lendres\textbackslash{}")
	\end{code}


Code folding:
https://jupyter-contrib-nbextensions.readthedocs.io/en/latest/install.html

\textcode{conda install -c conda-forge jupyter\_contrib\_nbextensions}
	\begin{figure}[tbp]
		\centering
		\includegraphics[width=\textwidth]{jupyterenablefolding}
		\caption[Enable nbextensions for Jupyter notebooks]{Enable nbextensions for Jupyter notebooks.}
		\label{fig:jupyterenablefolding}
	\end{figure}


	\section{Imbalanced}
	\begin{code}[\codenumbering]{}
		\codeitemnonumber \# Jupyter notebook
		\codeitemnonumber !pip install imblearn \switch{}user
		\codeitemnonumber !pip install imbalanced-learn \switch{}user
		\codeitemnonumber
		\codeitemnonumber \# Anaconda prompt
		\codeitemnonumber \#!pip install -U imbalanced-learn

		\codeitemnonumber \#conda install -c conda-forge imbalanced-learn

		\codeitemnonumber \# Restart the kernel after successful installation of the library
	\end{code}

	\section{Anaconda and Python}
	\begin{code}[\codenumbering]{}
		\codeitemnonumber conda update -n base conda
		\codeitemnonumber conda update \switch{}all
		\codeitemnonumber jupyter contrib nbextension install \switch{}user
	\end{code}

\textcode{conda install -c conda-forge sklearn-pandas}

	\section{Jupyter Notebooks}
	\subsection{Version}
\textcode{jupyter \switch{}version}

	\subsection{Original Start Up Short Cut}
Original short cut target value:
	\begin{code}[\codenumbering]{}
		\codeitemnonumber F:\textbackslash{}anaconda3\textbackslash{}python.exe F:\textbackslash{}anaconda3\textbackslash{}cwp.py F:\textbackslash{}anaconda3 F:\textbackslash{}anaconda3\textbackslash{}python.exe F:\textbackslash{}anaconda3\textbackslash{}Scripts\textbackslash{}jupyter-notebook-script.py "\%USERPROFILE\%/"
	\end{code}

\noindent Original start in value:
	\begin{code}[\codenumbering]{}
		\codeitemnonumber \%HOMEPATH\%
	\end{code}


	\subsection{Changing Jupyter Start Up Location}
This section shows how to change the default Jupyter start up folder.  This information was take from:
\href{https://stackoverflow.com/questions/35254852/how-to-change-the-jupyter-start-up-folder}{How to change the Jupyter start folder}

	\subsubsection{Jupyter Notebook and JupyterLab \texorpdfstring{$<$}{<} 3.0 Versions}
	\begin{numberedlist}
		\item Use \textcode{Run as Administrator} on the Anaconda Powershell Prompt.
		\item At the command prompt run:\\ \textcode{jupyter notebook \switch{}generate-config}. \label{nl:jupyterstartupfoldergenerateconfiguration}
		\item Open the written by the command in the previous step.  The file location is specified as output from the command.  Typically, it is This writes a file to:\\ \textcode{C:\textbackslash{}Users\textbackslash{}username\textbackslash{}.jupyter\textbackslash{}jupyter\_notebook\_config.py} \label{nl:jupyterstartupfolderconfigurationfile}
		\item Search for the following line in the file:\\ \textcode{\# c.NotebookApp.notebook\_dir = \textquoteright{}\textquoteright{}} \label{nl:jupyterstartupfoldersearchforline}
		\item Uncomment the line.
		\item Edit the path\footnote{\warning{Be sure to use forward slashes instead of back slashes in the path.}} to the desired location, for example:\\ \textcode{c.NotebookApp.notebook\_dir = "C:/Projects/Python/"}  \label{nl:jupyterstartupfoldereditpath}
		\item From the start menu, right click on the Jupyter Notebook tile and select:\\ \textcode{More->Open File Location}
		\item From the file explorer window that opened, right click on the Jupyter Notebook short cut and select \textcode{Properties}.
		\item In the \textcode{Target} text box, go to the end of the line and delete:\\ \textcode{"\%USERPROFILE\%/"}
		\item Click the \textcode{Apply} button.  If prompted that administrator rights are required, confirm that you want to continue.
		\item Click the \textcode{Ok} button.
	\end{numberedlist}

	\subsubsection{JupyterLab \texorpdfstring{$>=$}{>=} 3, Jupyter Notebook Classic, and RetroLab}
	\begin{bulletedlist}
		\item In step~\ref{nl:jupyterstartupfoldergenerateconfiguration} above, replace:\\ \textcode{jupyter notebook \switch{}generate-config}\\ with\\ \textcode{jupyter server \switch{}generate-config}.
		\item In step~\ref{nl:jupyterstartupfolderconfigurationfile} above, the file written is typically:\\ \textcode{jupyter\_server\_config.py}

		\item In steps~\ref{nl:jupyterstartupfoldersearchforline} and~\ref{nl:jupyterstartupfoldereditpath} above, the configuration line:\\ \textcode{c.NotebookApp.notebook\_dir}\\ is instead:\\ \textcode{c.ServerApp.root\_dir}
	\end{bulletedlist}


	\section{Imblearn library}
To install imblearn you can use:
	\begin{code}[\codenumbering]{}
		\codeitemnonumber !pip install imbalanced-learn==0.8.0 \# In Jupyter notebook
		\codeitemnonumber pip install imbalanced-learn==0.8.0 \# In anaconda prompt
	\end{code}


	\section{SMOTE}
I am getting this error while trying to import SMOTE, how do I fix it?

\noindent\textcode{ImportError: cannot import name `delayed' from `sklearn.utils.fixes' (C:\textbackslash{}Users\textbackslash{}anaconda3\textbackslash{}lib\textbackslash{}site-packages\textbackslash{}sklearn\textbackslash{}utils\textbackslash{}fixes.py)}

Use the following:
	\begin{code}[\codenumbering]{}
		\codeitemnonumber !pip install imbalanced-learn==0.8.0
		\codeitemnonumber !pip install delayed
	\end{code}


	\section{Yellowbrick}
\textcode{conda install -c districtdatalabs yellowbrick}
\textcode{python -m pip install yellowbrick}


	\section{Google Colab GPUs}
\textcode{Edit->Notebook Settings->Hardware Acceleration}

\textcode{!pip list}

from google.colab import drive
drive.mount("/content/drive")


	\section{TensorFlow}
\noindent CPU version:
	\begin{code}[\codenumbering]{}
		\codeitemnonumber conda create -n tf tensorflow
		\codeitemnonumber conda activate tf
		\codeitemnonumber conda install spyder
	\end{code}

\noindent GPU version:
	\begin{code}[\codenumbering]{}
		\codeitemnonumber conda create -n tf-gpu tensorflow-gpu
		\codeitemnonumber conda activate tf-gpu
		\codeitemnonumber conda install spyder
	\end{code}

	\begin{numberedlist}
		\item Open Anaconda Navigator.
		\item In top left corner you see Selector: ``Applications on: base(root).''
		\item Change: ``base root'' for ``Tensorflow'' it assumes that it was already installed based on link above
		\item Install Spyder.
	\end{numberedlist}


	\section{Spacy}
Installed it from the Anaconda Navigator Environments
	\begin{code}[\codenumbering]{}
		\codeitemnonumber !pip install spacy
		\codeitemnonumber !python -m spacy download en\_core\_web\_sm
	\end{code}%


	\section{Manim}
The \textit{conda} version did not work.  Used the instructions at:
	\begin{code}[\codenumbering]{}
		\codeitemnonumber https://docs.manim.community/en/stable/installation/windows.html
	\end{code}%
under the heading for \textit{Manual Installation}.

Namely,
	\begin{numberedlist}
		\item Create a new virtual environment using the procedure listed above.
		\item Download a compiled set of \textit{FFmpeg} binaries.
		\item Add the location of the binaries (inside of the \textit{bin} directory) to the \textit{PATH} environmental variable.
		\item Run:
		\begin{code}[\codenumbering]{}
			\codeitemnonumber python -m pip install manim
		\end{code}
		\item Check the environmental variable \textit{PATH} to ensure the MiKTeX path is included.  To find the MiKTeX path you can use WinEdt and go to:
		\begin{plainlist}
			\item \textit{Options$\rightarrow$Execution Modes...$\rightarrow$Console Applications}
		\end{plainlist}
		\item Restart the computer.
	\end{numberedlist}

	\section{Anaconda and Python Upgrades}
	\begin{numberedlist}
		\item To force Anaconda to upgrade.
		\item Run:
		\begin{code}[\codenumbering]{}
			\codeitemnonumber conda update --force conda
			\codeitemnonumber conda update anaconda
			\codeitemnonumber conda update conda
		\end{code}
		\item To install an upgraded Python to the current environment (\textcode{-c conda-forge} is optional).
		\begin{code}[\codenumbering]{}
			\codeitemnonumber conda install -c conda-forge python=3.11
		\end{code}
		\item To install into a different environment.
		\begin{code}[\codenumbering]{}
			\codeitemnonumber conda install -n \_env-name\_ -c conda-forge python=3.11
		\end{code}
		\item \important{[PREFERRED]} Use Anaconda to create a new environment and install a new version of Python to it:
		\begin{code}[\codenumbering]{}
			\codeitemnonumber conda create -n \_env-name\_ python=3.11 anaconda
		\end{code}
		\item Install will a specific version of python and libraries.
		\begin{code}[\codenumbering]{}
			\codeitemnonumber conda install -n py311 -c conda-forge python=3.11 spyder-kernels=2.4 scikit-learn=1.2.2 matplotlib
		\end{code}
	\end{numberedlist}
