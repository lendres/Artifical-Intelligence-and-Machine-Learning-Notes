	\printlistofnotationschapterheader{}

Every attempt has been made to prevent a variable from having multiple definitions.  However, where this does occur the meaning should be clear from the context.  Some variables that are used generically are not defined here.

	\begin{listofnotations}
		\addnotationtitle{Syntax}
			\addmathnotation{a}{Italic font indicates a scalar value.}
			\addmathnotation{\mbf{a}}{Bold font indicates a vector, matrix, or tensor.}
			\addmathnotation{\vec{\bullet}}{A unit vector and/or coordinate system axis.}
			\addmathnotation{\left| \bullet \right|}{The magnitude (length) of $\bullet$.}

			\addnotationtitle{Latin Variables}
	
			% C
			\addmathnotation{\coeffvariation}{Coefficient of variation.}

			% H
			\addmathnotation{\nullhypothesis}{Null hypothesis.}
	
	 		% S
			\addmathnotation{\samplestandarddeviation}{Sample standard deviation.}
	
			% V
			\addmathnotation{\variance}{Variance.}
	
			% X
			\addmathnotation{\median}{Sample median it is defined as $(n-1)/2$.}
			\addmathnotation{\samplemean}{Sample mean (average).}
			\addmathnotation{\xsmallest}{Smallest number in a sample.}
			\addmathnotation{\xlargest}{Largest number in a sample.}


		\addnotationtitle{Greek Variables}
			% Alpha
			\addmathnotation{\levelofsignificance}{Level of significance.}		
			% Beta
			% Gamma
			% Delta
			% Epsilon
			% Zeta
			% Eta
			% Theta
			% Iota
			% Kappa
			% Lambda
			% Mu
			\addmathnotation{\populationmean}{Population mean (average).}
			% Nu
			% Xi
			% Omicron
			% Pi
			% Rho
			% Sigma
			\addmathnotation{\populationstandarddeviation}{Population standard deviation.}
			% Tau
			% Upsilon
			% Phi
			% Chi
			% Psi
			% Omega

		%\addmathnotationtitle{Typeface/Notation}
			%\addmathnotation{}{}
			%\addmathnotationtext{}{}
			%\addmathnotationmathdesc{}{}

	\end{listofnotations} 