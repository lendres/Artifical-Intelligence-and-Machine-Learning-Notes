	\chapter{Machine Learning Pipeline and Hyperparameter Tuning}

	\resetquestioncounter{}
	\begin{qanda}
		\begin{question}
What is the difference between fit, fit\_transform, and transform?
		\end{question}

		\begin{answer}
The way we use fit and predict in regression, similarly for functions that transform the data - we have fit and transform:

\textbf{fit} - is used to fit parameters of the function.

\textbf{transform} - transforming the data using parameters fitted with the fit function.

\textbf{fit\_transform} - to first fit the parameters of the function and then transform the data.

\href{https://towardsdatascience.com/what-and-why-behind-fit-transform-vs-transform-in-scikit-learn-78f915cf96fe}{Explanation}
		\end{answer}
	\end{qanda}

	\begin{qanda}
		\begin{question}
How do you tune a model using train, test, and validation split?
		\end{question}

		\begin{answer}
The process is:
			\begin{numberedlist}
				\item Pick a combination of hyperparameters.
				\item Train a model using those hyperparameters.
				\item Find the model's performance on the validation test.
				\item Repeat this process for all combinations available.
				\item Choose the model with the best validation score, and find out the final (generalized) score on the test set.
			\end{numberedlist}
		\end{answer}
	\end{qanda}

	\begin{qanda}
		\begin{question}
How do you upgrade the Numpy library?
		\end{question}

		\begin{answer}
 Jupyter notebook: \textcode{!pip install numpy==1.20.3 --user}

\noindent or

\noindent Anaconda prompt: \textcode{pip install numpy==1.20.3 --user}
		\end{answer}
	\end{qanda}

	\begin{qanda}
		\begin{question}
			Tuning the model using grid search is taking a long time to run. What are the recommendations for improving performance?
		\end{question}

		\begin{answer}
Tuning a model using grid search usually takes a long time, you can try the following to get more insights

\noindent\textcode{grid\_cv = GridSearchCV(estimator=pipe, param\_grid=param\_grid, scoring=scorer, cv=5, n\_jobs = -1, verbose = 2)}

\noindent n\_jobs = -1 can speed up the tuning process by utilizing all the CPU cores.

\noindent verbose = 2 will give you the number of times the model has to be fit so that you will get an idea of how much time will it take
		\end{answer}
	\end{qanda}


	\begin{qanda}
		\begin{question}
I am getting the same performance with both GridSearchCV and RandomizedSearchCV.  How can I change this as this doesn't look practical to me?
		\end{question}

		\begin{answer}
Getting the same results is not incorrect, you might get the same results from both grid and random search.  However few things that can be checked in such cases are:
	\begin{bulletedlist}
		\item If the value of n\_iter is greater than the possible number of combinations of hyperparameters then you will get the same results from both.
		\item Check if you have passed the obtained value of hyperparameters while building the model.
	\end{bulletedlist}

Getting the same results is not incorrect, you might get the same results from both grid and random search.  However few things that can be checked in such cases are:

Getting the same results is not incorrect, you might get the same results from both grid and random search.  However few things that can be checked in such cases are:
		\end{answer}
	\end{qanda}


	\section{Pipeline}
What is a pipeline?
	\begin{bulletedlist}
		\item Almost always, we need to tie together many different processes that we use to prepare data for machine learning based model.
		\item It is paramount that the stage of transformation of data represented by these processes are standardized
		\item Pipeline class of sklearn helps simplify the chaining of the transformation steps and the model
		\item Pipeline, along with the GridsearchCV helps search over the hyperparameter space applicable at each stage
	\end{bulletedlist}

	\subsection{The Pipeline Process}
	\begin{numberedlist}
		\item Sequentially apply a list of transforms and a final estimator.
		\item Intermediate steps of the pipeline must be ``transforms'', that is, they must implement fit and transform methods.
		\item The final estimator only needs to implement fit.
		\item Helps standardize the model project by enforcing consistency in building testing and production.
	\end{numberedlist}

	\begin{bulletedlist}
		\item The pipeline object requires all the stages to have both ``fit()'' and ``transform()'' function except for the last stage when it is an estimator.
		\item The estimator does not have a ``transform()'' function because it builds the model using the data from previous step. It does not transform the data
		\item The transform function transforms the input data and emits transformed data as output which becomes the input to the next stage
		\item pipeline.fit() calls the fit and transform functions on each stage in sequence. In the last stage, if it is an estimator, only the fit function is called to create the model.
		\item The model become a part of the pipeline automatically pipeline.predict() calls the transform function at all the stages.
	\end{bulletedlist}