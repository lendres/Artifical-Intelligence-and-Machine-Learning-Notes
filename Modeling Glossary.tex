% All statistics glossary entries.

%\newglossaryentry{mdg:}
%{
%	type=model,
%    name=,
%    description={}
%}

\newglossaryentry{mdg:bagging}
{
	type=model,
    name=bagging,
    description={Constructing models in parallel.  Also called bootstrap aggregating}
}

\newglossaryentry{mdg:boosting}
{
	type=model,
    name=boosting,
    description={Constructing models in serial}
}

\newglossaryentry{mdg:bootstrapaggregating}
{
	type=model,
    name=bootstrap aggregating,
    description={See \gls{mdg:bagging}}
}

\newglossaryentry{mdg:coherencellm}
{
	type=model,
    name=coherence in LLMs,
    description={Coherence refers to the ability of the model to produce text that is consistent and logical. While coherence is important for generating meaningful responses, it does not specifically address the relevance of the responses to the user's query.}
}

\newglossaryentry{mdg:contextuallm}
{
	type=model,
    name=contextual in LLMs,
    description={Contextual behavior in LLMs refers to the model's ability to generate responses that are relevant to the context of the conversation or query. This behavior ensures that the responses provided by the model are appropriate and aligned with the user's query.}
}

\newglossaryentry{mdg:covariateshift}
{
	type=model,
    name=covariate shift,
    description={When the distribution of input data shifts between the training environment and live environment.  Although the input distribution may change, the output distribution or labels remain the same}
}

\newglossaryentry{mdg:curseofdimensionality}
{
	type=model,
    name=curse of dimensionality,
    description={When there are a large number of parameters but not enough data samples to accurately represent them all and the possible relationships between them}
}

\newglossaryentry{mdg:ensemble}
{
	type=model,
    name=ensemble techniques,
    description={Using multiple models to obtain better predictive performance}
}

\newglossaryentry{mdg:hallucinationllm}
{
	type=model,
    name=hallucination in LLMs,
    description={Hallucination behavior in LLMs refers to instances where the model generates inaccurate or nonsensical outputs, not relevant to the user's query. This behavior is opposite to generating responses that are relevant to the user's query.}
}

\newglossaryentry{mdg:unstable}
{
	type=model,
    name=unstable,
    description={Models that are very sensitive to small changes in the data (small changes in data lead to a different model)}
}

\newglossaryentry{mdg:unsupervisedlearning}
{
	type=model,
    name=unsupervised learning,
    description={Unsupervised learning uses machine learning algorithms to analyze and cluster unlabeled data sets. These algorithms discover hidden patterns in data without the need for human intervention (hence, they are ``unsupervised'').  Unsupervised learning models are used for three main tasks: clustering, association and dimensionality reduction}
}

\newglossaryentry{mdg:oddsratio}
{
	type=model,
    name=odds ratio,
    description={The probability of successful result divided by the probability of failure result}
}

\newglossaryentry{mdg:perceptron}
{
	type=model,
    name=perceptron,
    description={An algorithm for supervised learning of binary classifiers.  It is the fundamental building block of neural networks}
}

\newglossaryentry{mdg:randomforrest}
{
	type=model,
    name=random forrest,
    description={Bagging that uses decision trees as the models}
}

\newglossaryentry{mdg:rmsprop}
{
	type=model,
    name=RMSprop,
    description={Root mean squared propagation optimization technique}
}

\newglossaryentry{mdg:supervisedlearning}
{
	type=model,
    name=supervised learning,
    description={Supervised learning is when you already know the label (value) of the target variable. It is of two types: regression (for continuous variables) and classification (for categorical or discrete values)}
} 

\newglossaryentry{mdg:variabilityllm}
{
	type=model,
    name=variability in LLMs,
    description={Variability refers to the diverse range of outputs generated by the model. While variability is a characteristic of LLMs, it does not specifically address the model's ability to generate responses relevant to the user's query.}
}